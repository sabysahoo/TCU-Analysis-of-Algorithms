%%%%%%%%%%%%%%%%%%%%%%%%%%%%%%%%%%%%%%%%%%%%%%%%%%%%%%%%%%%%%%%%%%
%
% Analysis of Algorithms
%
% Homework Assignment #2
%
%%%%%%%%%%%%%%%%%%%%%%%%%%%%%%%%%%%%%%%%%%%%%%%%%%%%%%%%%%%%%%%%%%
%%%%%%%%%%%%%%%%%%%%%%%%%%%%%%%%%%%%%%%%%%%%%%%%%%%%%%%%%%%%%%%%%%
%
% Score Card and Answer Sheets
%
%%%%%%%%%%%%%%%%%%%%%%%%%%%%%%%%%%%%%%%%%%%%%%%%%%%%%%%%%%%%%%%%%%
\documentclass[addpoints,11pt]{exam}
\usepackage{clrscode4e}
\usepackage{tcucosc}
\usepackage{units}
\usepackage{enumitem}


%%%%%%%%%%%%%%%%%%%%%%%%%%%%%%%%%%%%%%%%%%%%%%%%%%%%%%%%%%%%%%%%%%
%
% Begin Document
%
%%%%%%%%%%%%%%%%%%%%%%%%%%%%%%%%%%%%%%%%%%%%%%%%%%%%%%%%%%%%%%%%%%
\begin{document}
\pagestyle{empty}


\noindent{\large\bfseries Name: YOUR NAME GOES HERE{\hrulefill}}\\
\noindent{\large\bfseries COSC 40403 - Analysis of Algorithms: Fall 2018: Homework 2}\\
\noindent{\large\bfseries Due: 23:59:59 on September 11, 2018}

%%%%%%%%%%%%%%%%%%%%%%%%%%%%%%%%%%%%%%%%%%%%%%%%%%%%%%%%%%%%%%%%%%
%
% Score Card and Answer Sheets
%
% Comment out one-or-the-other to show or not-show the answers.
%
%%%%%%%%%%%%%%%%%%%%%%%%%%%%%%%%%%%%%%%%%%%%%%%%%%%%%%%%%%%%%%%%%%
\printanswers
%%\noprintanswers


%%%%%%%%%%%%%%%%%%%%%%%%%%%%%%%%%%%%%%%%%%%%%%%%%%%%%%%%%%%%%%%%%%
%
% Score Card
%
%%%%%%%%%%%%%%%%%%%%%%%%%%%%%%%%%%%%%%%%%%%%%%%%%%%%%%%%%%%%%%%%%%
\ifprintanswers
\noindent
\begin{center}
	\gradetable[v][questions]
\end{center}
\newpage
\fi


%%%%%%%%%%%%%%%%%%%%%%%%%%%%%%%%%%%%%%%%%%%%%%%%%%%%%%%%%%%%%%%%%%
%
% Question 1
%
%%%%%%%%%%%%%%%%%%%%%%%%%%%%%%%%%%%%%%%%%%%%%%%%%%%%%%%%%%%%%%%%%%
\begin{questions}
\question[5]
Show that $f(n) = n^2 + 3n^3 \in \Theta(n^3)$.  That is, use the definitions of $O$ and $\Omega$ to show that $f(n)$ is both $O(n^3)$ and $\Omega(n^3)$. 
\begin{solutionorbox}

Big-O : O(g(n)) = \{ $f$(n) : there exist positive constants $c_1$ and $n_0$ such that \\
	0 $\leq$ $f$(n) $\leq$ $c_1$.g(n) for all n $\geq$ $n_0$ \}.\\
	
	$\hspace{5mm}$ 3$n^3$ + $n^2$ $\leq$ $c_1$.$n^3$ $\hspace{5mm}$ -----1-----  \\
	

Big-$\Omega$ : $\Omega$(g(n)) = \{ $f$(n) : there exist positive constants $c_2$ and $n_0$ such that \\
0 $\leq$ $c_2$.g(n) $\leq$ $f$(n) for all n $\geq$ $n_0$ \}.\\

	$\hspace{5mm}$ $c_2$.$n^3$ $\leq$ 3$n^3$ + $n^2$   $\hspace{5mm}$ -----2----- \\
	
Big-$\Theta$ : $\Theta$(g(n)) = \{ $f$(n) : there exist positive constants $c_1$,$c_2$ and $n_0$ such that \\
0 $\leq$ $c_2$.g(n)  $\leq$ $f$(n) $\leq$ $c_1$.g(n) for all n $\geq$ $n_0$ \}.\\

	$\hspace{5mm}$ From 1 $\&$ 2, we get: \\
	
  	$\hspace{5mm}$  $c_2$.$n^3$ $\leq$ 3$n^3$ + $n^2$ $\leq$ $c_1$.$n^3$ $\hspace{5mm}$ -----3-----\\
  	
	$\hspace{5mm}$  $c_2$ $\leq$ 3 + 1/n $\leq$ $c_1$ \\
	
	$\hspace{5mm}$  So, for n $\geq$ 1, $c_1$ $\geq$ 4 $\&$ $c_2$ $\leq$ 3 \\
	
	$\hspace{5mm}$ equation 3 always holds true and satisfies the conditions for Big-$\Theta$. \\ \\
	$\hspace{5mm}$ Hence, $f(n) = n^2 + 3n^3 \in \Theta(n^3)$. \\
		
\end{solutionorbox}

\ifprintanswers
\newpage
\else
\bigskip
\fi


%%%%%%%%%%%%%%%%%%%%%%%%%%%%%%%%%%%%%%%%%%%%%%%%%%%%%%%%%%%%%%%%%%
%
% Question 2
%
%%%%%%%%%%%%%%%%%%%%%%%%%%%%%%%%%%%%%%%%%%%%%%%%%%%%%%%%%%%%%%%%%%
\question[5]
Suppose you have a computer that requires 1 minute to solve problem instances of size $n=1000$.  Suppose you buy a new computer that runs 1000 times faster than the old one.  What instance sizes can be run in 1 minute, assuming the following time complexities $T(n)$ for our algorithm?
\begin{enumerate}[label=(\alph*)]
	\item $T(n) = n$
	\item $T(n) = n^3$
	\item $T(n) = 10^n$
\end{enumerate}
\begin{solutionorbox}
	The old computer could do 1000 instances per min. Now, the new one can do 1000*1000 instances per min
	\begin{enumerate}[label=(\alph*)]
		\item $T(n) = n$ --- 1,000,000 = n
		\item $T(n) = n^3$   $1000^3$*1000 = $10,000^3$ , hence n = 10,000
		\item $T(n) = 10^n$	 $10^{1000}$*1000 = $10^{1003}$ , hence n = 1003
	\end{enumerate}
\end{solutionorbox}

\ifprintanswers
\newpage
\else
\bigskip
\fi


%%%%%%%%%%%%%%%%%%%%%%%%%%%%%%%%%%%%%%%%%%%%%%%%%%%%%%%%%%%%%%%%%%
%
% Question 3
%
%%%%%%%%%%%%%%%%%%%%%%%%%%%%%%%%%%%%%%%%%%%%%%%%%%%%%%%%%%%%%%%%%%
\question[5]
Let $f(n)$ and $g(n)$ be asymptotically nonnegative functions.  Using the basic definition of $\Theta$-notation, prove that $\max(f(n),g(n)) = \Theta(f(n)+g(n))$.
\begin{solutionorbox}

 $Theorem$ $3.1$:\\
For any two functions $f(n)$ and $g(n)$, we have  $f(n) = \Theta(g(n))$, if and only if, $f(n) = O(g(n))$ and $f(n) = \Omega(g(n))$. \\

$\hspace{5mm}$ $f(n)$ $\leq$ $f(n)$ + $g(n)$ and $g(n)$ $\leq$ $f(n)$ + $g(n)$ \\

$\hspace{5mm}$ So, $\max(f(n),g(n)) = \O(f(n)+g(n))$ $\hspace{5mm}$ (Upper-Bound). --1-- \\

$\hspace{5mm}$ $f(n)$ + $g(n)$  $\leq$ $\max(f(n),g(n))$ \\

$\hspace{5mm}$ $\rightarrow$ 1/2($f(n)$ + $g(n)$)  $\leq$ $\max(f(n),g(n))$ \\

$\hspace{5mm}$ So, $\max(f(n),g(n)) = \Omega(f(n)+g(n))$ $\hspace{5mm}$ (Lower-Bound). --2-- \\

Hence, from $Theorem 3.1$, 1 $\&$ 2 we have proved that $\max(f(n),g(n)) = \Theta(f(n)+g(n))$.


\end{solutionorbox}

\ifprintanswers
\newpage
\else
\bigskip
\fi


%%%%%%%%%%%%%%%%%%%%%%%%%%%%%%%%%%%%%%%%%%%%%%%%%%%%%%%%%%%%%%%%%%
%
% Question 4
%
%%%%%%%%%%%%%%%%%%%%%%%%%%%%%%%%%%%%%%%%%%%%%%%%%%%%%%%%%%%%%%%%%%
\question[5]
Show that the golden ratio $\phi$ and its conjugate $\hat\phi$ both satisfy the equation $x^2 = x+1$.
\begin{solutionorbox}
	$\Phi$ = $\dfrac{1+\sqrt{5}}{2}$ $\hspace{5mm}$ $\&$ $\hspace{5mm}$	$\hat{\Phi}$ = $\dfrac{1-\sqrt{5}}{2}$ \\
	
	$\hspace{5mm}$ Now, $x^2 = x+1$ \\
	
	$\hspace{5mm}$ $\rightarrow$  $(\dfrac{1+\sqrt{5}}{2})^2$ - $\dfrac{1+\sqrt{5}}{2}$ - 1 = 0 \\
	
	$\hspace{5mm}$ $\rightarrow$  $\dfrac{1 - 2*\sqrt{5} + 5 - 2 + 2*\sqrt{5} - 4}{4}$ = 0 \\
	
	$\hspace{5mm}$ Hence proved.
	
	
\end{solutionorbox}

\ifprintanswers
\newpage
\else
\bigskip
\fi

%%%%%%%%%%%%%%%%%%%%%%%%%%%%%%%%%%%%%%%%%%%%%%%%%%%%%%%%%%%%%%%%%%
%
% Question 5
%
%%%%%%%%%%%%%%%%%%%%%%%%%%%%%%%%%%%%%%%%%%%%%%%%%%%%%%%%%%%%%%%%%%
\question[5]
Prove by induction that the $i$th Fibonacci number satisfies the equailty
$$F_i = \frac{\phi^i - \hat{\phi^i}}{\sqrt{5}}$$
where $\phi$ is the golden ratio and $\hat{\phi^i}$ is its conjugate.
\begin{solutionorbox}
	$Proof: $ \\
	
	$\hspace{5mm}$ For n = 0, 
	
	$\hspace{5mm}$ $\hspace{5mm}$ $\hspace{5mm}$ 0 = $\dfrac{1 - 1}{\sqrt{5}}$ \\ 
	
	$\hspace{5mm}$ For n = 1, 	
	
	$\hspace{5mm}$ $\hspace{5mm}$ $\hspace{5mm}$ 1 = $\dfrac{(1+\sqrt{5})^2 - (1-\sqrt{5})^2}{\sqrt{5}.2}$ \\ 
	
	$\hspace{5mm}$ Assume that $$F_i = \frac{\phi^i - \hat{\phi^i}}{\sqrt{5}}$$
	
	$\hspace{5mm}$ holds true for n-1 and n. We need to show that 
	
	$$F_{n+1} = \frac{\phi^{n+1} - \hat{\phi^{n+1}}}{\sqrt{5}}$$
	
	$\hspace{5mm}$ We know from Fibonacci series,
	
	$$F_{n+1} = F_{n} + F_{n-1}$$
	
	$$F_{n+1} = \frac{\phi^{n} - \hat{\phi^{n}}}{\sqrt{5}} + \frac{\phi^{n-1} - \hat{\phi^{n-1}}}{\sqrt{5}}$$
	
	$$F_{n+1} = \frac{\phi^{n} - \hat{\phi^{n}} + \phi^{n-1} - \hat{\phi^{n-1}}}{\sqrt{5}}$$
	
	$$F_{n+1} = \frac{\phi^{n} + \phi^{n-1} - \hat{\phi^{n}} - \hat{\phi^{n-1}}}{\sqrt{5}}$$
	
	$\hspace{5mm}$ Since $\phi$.$\hat{\phi}$ = -1 \\
	
	$$F_{n+1} = \frac{(\hat{\phi^{2}} - \hat{\phi^{1}})\phi^{n+1} + ({\phi^{2}} - {\phi^{1}})\hat{\phi^{n+1}}}{\sqrt{5}}$$
	
	$\hspace{5mm}$ Since $(\hat{\phi^{2}} - \hat{\phi^{1}})$ = $({\phi^{2}} - {\phi^{1}})$ = 1 $\hspace{45mm}$  \{ from Question 4 \} \\
	
	$$F_{n+1} = \frac{\phi^{n+1} - \hat{\phi^{n+1}}}{\sqrt{5}}$$
	
	$\hspace{5mm}$ Hence proved by induction.
	
\end{solutionorbox}

\ifprintanswers
\newpage
\else
\bigskip
\fi


%%%%%%%%%%%%%%%%%%%%%%%%%%%%%%%%%%%%%%%%%%%%%%%%%%%%%%%%%%%%%%%%%%
%
% Question 6
%
%%%%%%%%%%%%%%%%%%%%%%%%%%%%%%%%%%%%%%%%%%%%%%%%%%%%%%%%%%%%%%%%%%
\question[5]
Consider the following algorithm:
\begin{codebox}
	\Procname{$\proc{Print-I-J}(n)$}
	\li \For $(i=2; i<n; i++)$ \Do
	\li 	\For $(j=0; j\le n)$ \Do
	\li 		\proc{Print} $i$, $j$
	\li 		$j = j + \floor{n/4}$
	\End
	\End
	\End
\end{codebox}

\begin{enumerate}[label=(\roman*)]
	\item What does this algorithm do?
	\item What is the output when $n=4$, $n=16$, $n=32$?
	\item What is the time complexity $T(n)$.  You may assume that $n$ is divisible by 4.
\end{enumerate}

\begin{solutionorbox}
	
	\begin{enumerate}[label=(\roman*)]
		\item The algorithm prints out pairs of \{i,j\} where i takes values from \{2 to n\} and j takes values in a geometric progression with the first value as 0, the ratio of $\dfrac{n}{4}$ and the last value as n.
		\item \begin{minipage}[b]{0.30\textwidth}
			\raggedright
			For  n = 4,\\
			loop i = 2\\
			i - 2, j - 0\\
			i - 2, j - 1\\
			i - 2, j - 2\\
			i - 2, j - 3\\
			i - 2, j - 4\\
			loop i = 3\\
			i - 3, j - 0\\
			....... \\
			i - 3, j - 4\\
			loop i = 4\\
			i - 4, j - 0\\
			i - 4, j - 1\\
			i - 4, j - 2\\
			i - 4, j - 3\\
			i - 4, j - 4\\ 
			
		\end{minipage}%
		\begin{minipage}[b]{0.30\textwidth}
			\centering
			For  n = 16,\\
			loop i = 2\\
			i - 2, j - 0\\
			i - 2, j - 4\\
			i - 2, j - 8\\
			i - 2, j - 12\\
			i - 2, j - 16\\
			loop i = 3\\
			i - 3, j - 0\\
			....... \\
			i - 15, j - 16\\
			loop i = 16\\
			i - 16, j - 0\\
			i - 16, j - 4\\
			i - 16, j - 8\\
			i - 16, j - 12\\
			i - 16, j - 16\\
		\end{minipage}%
		\begin{minipage}[b]{0.30\textwidth}
			\raggedleft
			For  n = 32,\\
			loop i = 2\\
			i - 2, j - 0\\
			i - 2, j - 8\\
			i - 2, j - 16\\
			i - 2, j - 24\\
			i - 2, j - 32\\
			loop i = 3\\
			i - 3, j - 0\\
			....... \\
			i - 31, j - 32\\
			loop i = 32\\
			i - 32, j - 0\\
			i - 32, j - 8\\
			i - 32, j - 16\\
			i - 32, j - 24\\
			i - 32, j - 32\\
		\end{minipage}
					
					Attached python file to check the outputs.
		
		
		\item  Algorithm:\\
		
		\begin{lstlisting}
n = 32				c1		 1
for i in range(2, n+1):		c2		 n
  j = 0				c3	 	 n-1
  while (j < n + 1):		c4		(n-1)*6
    print(i, j)			c5 		(n-1)*5	
    j = j + math.floor(n/4)	c6		(n-1)*5
		\end{lstlisting}
		
		T(n) = c1*1 + c2*n + c3*n-1 + c4*(n-1)*6 + c5*(n-1)*5 + c6*(n-1)*5
		T(n) = $\theta(n)$
	
	\end{enumerate}
	
\end{solutionorbox}

\ifprintanswers
\newpage
\else
\bigskip
\fi

\end{questions}
\end{document}
